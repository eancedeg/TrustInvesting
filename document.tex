\documentclass[12pt,letterpaper]{article}
\usepackage[utf8]{inputenc}
\usepackage[T1]{fontenc}
\usepackage[spanish]{babel}
\usepackage{amsmath}
\usepackage{amsfonts}
\usepackage{amssymb}
\usepackage{graphicx}
\usepackage[left=2.50cm, right=2.50cm, top=2.00cm, bottom=2.00cm]{geometry}
\usepackage[breaklinks,colorlinks=true,linkcolor=blue,citecolor=blue,urlcolor=blue]{hyperref}
\usepackage{pdfpages}
\usepackage{enumitem}
\setlist[enumerate]{itemsep=2pt,topsep=4pt}
\setlist[itemize]{itemsep=2pt,topsep=4pt}
\setlist[description]{itemsep=2pt,topsep=4pt}

\usepackage{tocloft}
\renewcommand{\cftsecdotsep}{\cftdotsep}


\renewcommand{\baselinestretch}{1.5}
\renewcommand{\labelitemi}{$\bullet$}
\begin{document}
	
	\begin{titlepage}
		\includepdf{portada} 
	\end{titlepage}


	\tableofcontents
	\thispagestyle{empty}
	\newpage
	
	\section*{Introducción}
	\addcontentsline{toc}{section}{Introducción}
	\setcounter{page}{1}
	
	Muchas personas se preguntan si existen formas de utilizar los ahorros que se acumulan por años para que el dinero no se encuentre estático en cuentas bancarias, teniendo en cuenta sobre todo las escasas ganancias que se obtienen por parte de las instituciones bancarias. Atendiendo a lo anterior muchas personas han optado por recibir cursos de trading, arbitraje, etc. para aprender a invertir sus ahorros en la bolsa directamente. Estos métodos tienen el problema de que al principio y con poca experiencia, es fácil perder en muchas ocasiones el dinero.
	
	Por otro lado, desde el 2009 se viene hablando de las criptomonedas o divisas alternativas, que no son más que monedas digitales que se utilizan como medios de pago alternativos. En el año 2009 nace el \textbf{Bitcoin (BTC)} como la primera criptomoneda. Este tipo de criptodivisas tienen características muy interesantes:
	
	\begin{description}
		\item[Descentralización:] Esta moneda no es controlada por ningún país o banco en específico, por lo que brinda privacidad y mucha facilidad al realizar transacciones internacionales
	\end{description}
	
	principales que distinguen a este tipo de divisas son que son descentralizadas, esto significa que no son controladas por ningún banco o gobierno específicamente

	Se dice que la primera transacción de criptomonedas en Cuba se hizo desde un parque wifi de Centro Habana en julio de 2015. Este nuevo tipo de mercado ha tomado auge en nuestra isla con el aumento de la posibilidad de acceso a Internet para la población y ahora ya se pueden comercializar dentro del territorio nacional por medio de nuevas compañías como lo es Trust Investing, que además de proporcionar facilidades para el manejo de divisas, permite incrementar ese dinero mediante la compra de un paquete de inversión.
	
	El objetivo de este documento es describir brevemente el trabajo que realiza esta compañía y transmitir los beneficios con los cuales podemos hacernos los cubanos.
	
	
	
	\newpage
	\section{¿Qué es Trust Investing?}
	Trust Investing es una compañía registrada legalmente en Panamá, especializada en la gestión de inversiones financieras, basadas en operaciones reales de Forex, Trading y Arbitraje en criptomonedas, siguiendo el Network Marketing como modelo de negocio. Su Corporativo está integrado por: Diego Chaves como CEO, Fabiano de Lima como Director de Marketing y Claudio Barbosa como Director de Tecnología. Desde el 7 de mayo de 2019 cuando salió al mercado, Trust Investing estableció su misión de hacer del mercado financiero un canal práctico en el que las personas puedan invertir y asociarse a su Club de Beneficio, el que entrará en vigor cuando la compañía cumpla 5 años en el mercado o haya afiliado a un millón de personas o alcance mil millones de dólares en capital.
	
	\section{¿Qué es el trading y el arbitraje?}
	Trading es una palabra inglesa que significa compraventa y se basa en obtener beneficios a través de la compra y la venta a partir de estudiar el mercado, realizar análisis estadísticos y basarse en la experiencia. Por otra parte, el arbitraje es un término que proviene de las ciencias económicas y financieras y se refiere a la práctica de realizar operaciones de compraventa de manera casi simultáneas de un activo en diferentes mercados financieros y casas de cambio de criptomonedas, para sacar provecho de la diferencia de precios.
	
	Si comparamos el Trading y el Arbitraje en cuanto a Riesgo, Conocimiento necesario, Rapidez en las ganancias y Dificultad en su realización; en el caso del Trading, conlleva un mayor riesgo, mayor conocimiento, implica invertir mayor tiempo para obtener ganancias y es más difícil de realizar que el Arbitraje.
	
	\section{¿Es Trust Investing una empresa sostenible?}
	Trust Investing aplica la fórmula de 60 – 40 a toda la inversión que realizan sus afiliados. ¿Qué quiere decir esto? Que, si un afiliado adquiere un plan de inversión de 100, el 60 \% de esta cantidad es utilizado para las acciones de trading y arbitraje en los mercados financieros y casas de cambio de criptomonedas y el otro 40 \%, es utilizado para la expansión de la compañía a través del pago de incentivos a sus afiliados por la formación de equipos, como son: el Bono de Inicio Rápido y el Bono Binario.
	Por tal motivo, cuando un afiliado realiza una inversión, ésta no comienza a rendir automáticamente, sino después de un plazo de 48 horas; tiempo en el cual Trust Investing pone esta inversión en la mano de los traders mediante contrato, quienes mediante sus acciones permiten el pago de las cuotas diarias de lunes a viernes con espacio de 21 días.
	
	\section{¿Qué planes de inversión ofrece Trust Investing?}
	Trust Investing pone a disposición de sus afiliados un conjunto de planes de inversión que van desde los 15 dólares hasta los 100 000 dólares. Estos planes de inversión se dividen en tres clases fundamentales:
	
	\begin{itemize}
		\item La clase Prime contiene a los planes de 15, 30, 60, 100, 200 dólares.
		\item La clase Select contiene a los planes de 300, 500, 1000 y 2000 dólares.
		\item La clase Black contiene a los planes de 5 000, 10 000, 20 000, 50 000 y 100 000 dólares.
	\end{itemize}

Cada plan de inversión posee un nombre, un valor, un retorno final y un límite diario o techo binario. Para ejemplificar lo dicho anteriormente, tomaré como referencia el plan menor y el plan mayor. El plan menor que ofrece la compañía lleva por nombre Star, su valor es de 15 dólares, posee un retorno final de 30 dólares y su límite diario o techo binario es de 15 dólares. El plan mayor se denomina Eleven Stars, posee un valor de 100 000 dólares, un retorno final de 200 000 dólares y un techo binario de 10 000 dólares.

El retorno final siempre es el doble del valor del plan, pues la compañía se compromete a duplicar la inversión de sus afiliados a través de las acciones del trading; es decir, alcanzar el 200 \% del capital invertido.

\section{Ganancias reales de lunes a viernes}
Trust Investing asegura ganancias reales entre un 0,1 y un 5 \% en base a la inversión realizada de cada afiliado, siempre de lunes a viernes, días hábiles, porque la compañía trabaja con Forex, el mercado internacional de divisas más grande del mundo que realiza operaciones 24 horas al día de lunes a viernes.

El nombre Forex proviene de la unión de las palabras inglesas Foreign y Exchange. Para solo tener una idea de por qué Forex es el mayor mercado de divisas a nivel mundial, es válido decir, que se mueven diariamente en sus transacciones entre 5 y 6 trillones de dólares.

\section{¿Cómo funciona el Bono de Inicio Rápido?}
El Bono de Inicio Rápido, antiguamente conocido como Indicación Directa, no es más que una forma dentro del plan de marketing de la compañía, para incentivar la formación de equipo. Consiste en recibir el 10 \% de la cantidad invertida por cada referido directo al adquirir algún plan de inversión.

Pongamos un ejemplo: Juan invita a Pedro a formar parte de Trust Investing y Pedro decide adquirir como su primer plan de inversión el Three Stars, de 100 dólares. Juan recibe inmediatamente 10 dólares por el Bono de Inicio Rápido; este dinero pasa a formar parte del disponible para retirar de Juan, también, inmediatamente. En el caso que el referido directo realice no una nueva compra, sino una renovación de su plan de inversión, el Inicio Rápido bonifica con el 5 \%. Siguiendo el ejemplo anterior: el plan Three Stars de Pedro alcanzó por las acciones del trading de la compañía, un rendimiento del 200 \%, es decir, duplicó. En este momento Pedro decide renovar su plan de inversión con el mismo Three Stars. Juan, su patrocinador, recibe inmediatamente no 10 dólares, sino 5, el 5 \% por el acto de renovación. Los planes también se deben renovar si alcanzan un 300 \% de rendimiento por la formación de equipo.

\section{¿Cómo funciona el bono binario?}
El Bono Binario, no es más que una forma dentro del plan de marketing de la compañía, para incentivar la formación de equipo. Cada afiliado tiene la posibilidad de formar dos equipos: uno a su derecho y otro a su izquierda. El lado binario por el que afiliado es invitado constituye su lado fuerte o de derrame y el lado contrario, constituye su lado débil o de trabajo. Para activar el Bono Binario, es necesario que el afiliado invite a su primer referido directo por el mismo lado en que él fue invitado, es decir, por su lado fuerte, y que este invitado adquiera su primer plan de inversión. El segundo paso consiste en invitar al segundo referido directo por el otro lado, es decir, por el lado débil o de trabajo, y que este segundo invitado adquiera su primer plan de inversión. En este momento, se activará el Bono Binario de este afiliado y podrá comenzar a cobrar, una vez al día, por este concepto. ¿Cómo funciona? Veámoslo a través de un ejemplo: Juan fue invitado por el lado izquierdo, por lo que la izquierda constituye su lado fuerte o de derrame. Con el objetivo de activar su bono binario, Juan invita a Pedro, su primer referido directo, por el mismo lado por el que le invitaron a él, el izquierdo. Pedro adquiere su primer plan de inversión de 100 dólares. Juan necesita referir a alguien más para activar su Bono Binario, por lo que invita a María por su lado débil o de trabajo y esta adquiere su primer plan de inversión de 60 dólares.

Una vez activado el Bono Binario de Juan, este comienza a ganar por este concepto siguiendo el siguiente algoritmo:

\begin{enumerate}
	\item Se contabiliza el volumen de ventas o la cantidad de dólares invertidos por los equipos de ambos lados; en este caso, 100 por la izquierda y 60 por la derecha.
	\item Se recibe el 10\% de la cantidad de dólares invertidos por el equipo de menor inversión; en este caso, 6 dólares por el equipo de la derecha.
	\item Se resta ese total invertido (\$60) de la cantidad de ambos lados; por lo que quedarían 40 dólares por la izquierda y 0 dólares por la derecha. Para continuar recibiendo pagos por este concepto, Juan necesita ampliar su red, invitando, en este caso, a nuevos referidos directos por su lado derecho que quedó en 0. Pero no se trata solo de Juan, sino también del trabajo de sus equipos de la izquierda y de la derecha, quienes le aportarán puntos binarios según vayan creciendo.
\end{enumerate}

\section{¿Qué es el techo binario y qué implicación tiene?}
El techo binario o límite diario, es una propiedad que posee cada plan de inversión de los disponibles en la compañía que, como su nombre lo indica, solo afecta al Bono Binario. 

El techo binario del plan de inversión de 15 dólares es de 15 dólares; el del plan de 30, es de 30 dólares; el de 60, es 60 dólares y así sucesivamente; pero, ¡cuidado!, los planes de 5 000, 10 000, 20 000, 50 000 y 100 000, poseen límites diarios inferiores al valor del plan en sí. El límite diario del plan de 5 000 es de 3 000 dólares; el de 10 000, es de 4 000; el de 50 000, es 5 000 y el de 100 000 es de 10 000 dólares. 

El techo binario actúa como un regulador establecido en el plan de marketing de la compañía, que limita la cantidad máxima que puede cobrar un afiliado diariamente por el Bono Binario. También es una forma de incentivar la adquisición de planes de inversión superiores, teniendo en cuenta que, a mayor plan, mayor techo binario. Pero pongamos un ejemplo: Pedro posee varios planes de inversión, pero el más alto es un plan de 30 dólares, por lo que su techo binario es de 30. El excelente funcionamiento de su red le ha permitido contar con 1 000 puntos disponibles en su pierna fuerte y en un solo día, acumular 500 puntos en su pierna débil. En este caso, el algoritmo binario solo le podrá pagar 30 dólares al finalizar el día, aunque sí les descontará 500 puntos a su derecha y a su izquierda. Pedro dejó de ganar 20 dólares por binario. 

¿Consejo? Escalar a planes superiores que permitan aumentar el techo binario y poder aprovechar así los beneficios de la red.

\section{¿Cuándo se vence un plan de inversión?}
Un plan de inversión vence y, por lo tanto, necesita renovarse, cuando ocurre una de las siguientes dos condiciones:
\begin{description}
	\item[Condición No. 1:] haber alcanzado el 200 \% de rendimiento por las acciones de trading de la compañía; es decir, haber duplicado la inversión. 
	\item[Condición No. 2:] haber alcanzado el 300 \% de rendimiento por la formación de equipo que incluye dos bonificaciones, el de Inicio Rápido y el Bono Binario.
\end{description}
  
  \section{¿Cuál es la diferencia entre adquisición y renovación de un plan en Trust Investing?}
  Para entender la diferencia entre renovación y nueva adquisición de planes, es necesario saber que en Trust Investing solo existen dos balances independientes: las ganancias por el rendimiento del trading y las ganancias por la formación de equipo, que incluye el Bono de Inicio Rápido y el Bono Binario.
  
  La renovación de un plan solo debe realizarse cuando se haya cumplido una de las dos condiciones siguientes:
  
  \begin{enumerate}
  	\item El plan haya alcanzado el 200 por-ciento por el rendimiento del trading; es decir, que haya duplicado por este concepto.
  	\item El plan haya alcanzado el 300 por-ciento por el concepto de formación de equipo; es decir, que haya triplicado. En esta segunda condición, es válido aclarar que siempre se basa en el plan de inversión de mayor valor que haya adquirido el afiliado, en caso de poseer varios.
  \end{enumerate}
   
   De aquí se desprende que, un afiliado puede adquirir tantos planes de inversión como desee, incluso, del mismo valor; y a esto se le denomina adquisición de un nuevo plan. Sucede cuando aún sin llegar a su vencimiento por una de las dos condiciones anteriores, el afiliado decide adquirir uno nuevo de valor igual o distinto a los que ya poseía.
   
   \section{Los retiros en Trust Investing}
   En Trust Investing existen dos saldos contables independientes para cada afiliado, uno es el relacionado con el rendimiento de los paquetes de inversión por las acciones del trading de la compañía y el otro, por la formación equipo. 
   
   En el saldo contable por las acciones del trading, cada paquete de inversión adquirido, se contabiliza de manera independiente en plazos de 30 días; esto quiere decir, que el saldo acumulado por el rendimiento de cada paquete, se libera y queda disponible para retirar cada 30 días naturales, contados a partir de su adquisición. 
   
   Por otro lado, el saldo contable por la formación de equipo, que implica la bonificación por Inicio Rápido y el Binario, son liberados luego de adquiridos y pasan al disponible para retirar inmediatamente. 
   
   La compañía no establece un máximo de retiro y el mínimo, es de un dólar. A cada retiro solicitado se le aplica un arancel o comisión de un 5 \%.
   
   \section{Los planes de carrera en Trust Investing}
   Dentro del plan de marketing de Trust Investing, se contempla un incentivo de bonificación para aquellos grandes inversionistas y formadores de equipo a través de los planes de carrera. El plan de carrera en Trust Investing está constituido por 7 graduaciones que, de mayor a menor serían: Team Leader o Líder de Equipo, Manager o Director, Regional Director o Director Regional, National Director o Director Nacional, International Director o Director Internacional, Continental Director o Director Continental y Global Director o Director Global.
   
   Cada una de estas graduaciones poseen sus requisitos que pueden ser consultadas en el Dossier Informativo de la compañía, y bonifican mensualmente entre 100 y 150 000 dólares.
   
   \section{Quiero ser Team Leader}
   Team Leader es la primera graduación existente dentro de los planes de carrera de Trust Investing. Para alcanzar esta graduación, se debe cumplir con las siguientes condiciones sin importar el orden en que se obtengan:
   
   \begin{itemize}
   	\item Adquirir el plan Six Stars, de 1 000 dólares.
   	\item Contar con dos referidos directos que hayan adquirido el plan Six Stars, de 1000 dólares. No importa si estos referidos directos se encuentran ubicados en el lado derecho, izquierdo o el mismo lado del árbol binario.
   	\item Cada uno de esos directos, debe poseer 8 000 puntos o más en sus estructuras. Esto significa que la suma de los puntos históricos a la derecha con los puntos históricos a la izquierda de cada uno debe dar como resultado 8 000 o superior.
   \end{itemize}
        
   La graduación de Team Leader ofrece un bono mensual de 100 dólares.
   
   \section{¿Cómo puedo unirme?}
   Para comenzar a formar parte de la compañía debes registrarte a partir de algún usuario que ya tenga invertido en la misma a través de su enlace de referencia. Si no tiene ningún enlace de referencia a su disposición pude unirse a partir del que aparece más adelante.
   
   
\end{document}